% fastcap user's guide root latex file
\documentstyle[12pt]{article}

\pagestyle{headings}


\setlength{\oddsidemargin}{0.25in}      % 1.25in left margin 
\setlength{\evensidemargin}{0.00in}     % 1.00in left margin (even pages)
\setlength{\topmargin}{0.0in}           % 1in top margin
\setlength{\textwidth}{6.25in}          % 6.25in text
\setlength{\textheight}{9in}            % Body ht for 1in margins
\addtolength{\topmargin}{-\headheight}  % No header, so compensate
\addtolength{\topmargin}{-\headsep}     % for header height and separation

\newcommand{\thecode}{{\tt fftcap }}
\newcommand{\fastcap}{{\tt fastcap }}


\begin{document}

\begin{titlepage}
\vskip 36pt
\begin{center}
{\bf fftcap User Notes}
\end{center}

\vskip 18pt
\begin{center}
\hspace*{1.0in} J.~R.~Phillips \hfill J.\ White\hspace*{1.0in}
\end{center}

\vskip 18pt
\begin{center}
Research Laboratory of Electronics \\
Department of Electrical Engineering and Computer Science \\
Massachusetts Institute of Technology \\
Cambridge, MA  02139 U.S.A.
\end{center}
\vskip 18pt
\begin{center}
11 Mar 1996
\end{center}

\vfill
\noindent This work was supported by the Defense Advanced Research Projects
Agency contract N00014-91-J-1698,
the National Science Foundation contract (MIP-8858764 A02), an NDSEG Fellowship,
and grants from Digital Equipment Corporation, IBM and the Consortium for
Superconducting Electronics. 

\mbox{}
\vfill
\noindent Copyright \copyright\ 1992 Massachusetts Institute of Technology,
Cambridge, MA.  All rights reserved.

\vspace{\baselineskip}
\noindent This Agreement gives you, the LICENSEE, certain rights 
and obligations.
By using the software, you indicate that you have read, understood, and
will comply with the terms.

\vspace{\baselineskip}
\noindent M.I.T.\ MAKES NO REPRESENTATIONS OR WARRANTIES, 
EXPRESS OR IMPLIED.  By
way of example, but not limitation, M.I.T.\ MAKES NO REPRESENTATIONS OR
WARRANTIES OF MERCHANTABILITY OR FITNESS FOR ANY PARTICULAR PURPOSE OR
THAT THE USE OF THE LICENSED SOFTWARE COMPONENTS OR DOCUMENTATION WILL
NOT INFRINGE ANY PATENTS, COPYRIGHTS, TRADEMARKS OR OTHER RIGHTS.
M.I.T.\ shall not be held liable for any liability nor for any direct,
indirect or consequential damages with respect to any claim by LICENSEE
or any third party on account of or arising from this Agreement or use
of this software.

\end{titlepage}

\pagenumbering{arabic}
\setcounter{page}{1}

\section{Introduction}

This a short guide to \thecode, a three-dimensional capacitance
extraction program.  \thecode computes self and mutual capacitances
between ideal conductors of
arbitrary shapes, orientations and sizes.  

\thecode is built on top of the capacitance extraction code FASTCAP, and
thus the usage and capabilities of the two codes are similar.  You should
obtain the documentation for FASTCAP, in particular the FASTCAP Users
Guide, for full information on compilation, usage, and input file formats.
This note is devoted to describing some of the salient differences in the
usage and capabilities of the two codes. 

The algorithm used in
FastCap is an acceleration of the boundary-element technique for
solving the integral equation associated with the multiple-conductor,
multiple-dielectric
capacitance extraction problem.  
The linear system resulting  from the boundary-element discretization
is solved using a
a generalized conjugate
residual algorithm with a fast multipole algorithm to efficiently
compute the iterates.

\thecode differs from FASTCAP primarily in the way the iterates are
computed.  The fast-multipole algorithm of FASTCAP has been replaced with
a grid-based, ``precorrected-FFT'' algorithm.   For many geometries, the
grid-based algorithm is faster and more memory-efficient than the
fast-multipole algorithm implemented in FASTCAP.  More importantly, the
grid-based algorithm easily incorporates boundary-element analysis based on
modified Green functions.  A simple example of analysis using modified
Green functions, capacitance extraction in the presence of a planar
interface, is implemented in \thecode.    The planar interface may
be a groundplane, symmetry plane, or dielectric interface.  Currently the
analyis requires that all conductors must lie in the same half-space. 

In fact, the primary purpose of this release is to give an example of a
research implementation of the grid-based approach.  As such, \thecode is
not as flexible or robust as FASTCAP.  In particular, no preconditioner has
been implemented in \thecode, and no allowance is made for non-planar or
multiple dielectric interfaces.  However, for many problems, the algorithms
used in \thecode are more efficient than those used in FASTCAP, sometimes
by substantial factors. 

The algorithms of \thecode are discussed in 

\begin{enumerate}
\item J.~R.~Phillips and J.~White, ``Efficient Capacitance Extraction of 3D
  Structures Using Generalized Pre-corrected FFT methods,''  {\sl
    Proceedings of the IEEE 3rd Topical Meeting on Electrical Performance 
    of Electronic Packaging}, Monterey, CA, November 1994.

\item J.~Phillips and J.~White, ``A Precorrected-FFT method for Capacitance
  Extraction of Complicated 3-D Structures,'' {\sl Proceedings of the Int. Conf. on
    Computer-Aided Design}, Santa Clara, California, November 1994.

\item J.~R.~Phillips,  ``Error and complexity analysis for a
  collocation-grid-projection plus precorrected-{FFT} algorithm for solving
  potential integral equations with Laplace or Helmholtz kernels'' {\sl
 Proceedings of the 1995 Colorado Conference on Multigrid Methods}, Copper
Mountain, CO, April 1995. 

\end{enumerate}


\section{Running}

\subsection{Command Line Options}

The basic form of the \thecode program command line is
\begin{quote}
\begin{verbatim}
fftcap [-G<grid order>] [-d<depth>] [<input file>]
        [-z<reference level>] [-R<coef>] 
        [-l<list file>] [-t<iter tol>]
\end{verbatim}
\end{quote}

The primary run-time differences between \thecode and \fastcap are the {\tt
  -G,-d,-z,-R} flags. 

\begin{itemize}

\item {\tt -G <grid order>}. 

  The grid order variable controls the number of point charges used to
  approximate the long-range potential of charged panels. Accuracy
  generally increases with order, as does computational cost.  The default
  grid order is 3.  Permitted orders are 2,3,4 and 5.  In terms of the
  computed capacitance coefficients, grid order 2 is accurate to about $5 \times 10^{-2}$,
  order 3 to $5 \times 10^{-3}$, order 4 to  $10^{-3}$, and order 5 to
  $10^{-4}$.  The concept of grid order is the \thecode equivalent to
  multipole expansion order in FASTCAP. 

\item {\tt -d<depth>}. 

The size of the grid, that is the number of grid points throughout the
problem domain, is by default set automatically by a routine which attempts
to estimate the computationally optimal grid size.   The grid size may be
set by hand by using the {\tt -d} option.  If size $m$ is specified, {\tt
  -d$m$} then the code will allocate $2^m$ grid points along the longest
axis of the parallelipiped containing the problem domain.  Again, this is
similar to, yet different than, the FASTCAP ``depth'' variable. 

\item {\tt -z<reference level>} 

\thecode has the capability to extract the capacitance coefficients of a
system of conductors in the presence of a planar interface.  When the
argument {\tt -z$z_0$} is given, the code assumes that the interface is
located at $z$-coordinate $z_0$, in the coordinate system of the input
conductors.  All the conductors must have $z$-coordinates greater than
$z_0.$ 

The compile flag
{\tt GNDPLN} must be set {ON} to enable analysis of problems with planar
interfaces. 

\item {\tt -R<coef>} 

This flag specifies the nature of the planar interface.  Essentially, {\tt
  <coef>} is the relative magnitude of the image charge used to construct
the modified Green function. The compile flag
{\tt GNDPLN} must be set {ON}.  For dielectric interfaces, the coefficient
$R$ is given by 

\[ R = \frac{\epsilon_1 - \epsilon_2}{\epsilon_1 + \epsilon_2} \] 

where $\epsilon_1$ is the dielectric constant of the space the conductors
are imbedded in, and $\epsilon_2$ is the dielectric constant of the other
half-space. 

For ground planes,  {\tt -R-1} is used.  For symmetry planes, ${\tt -R1}$
is used. 

For problems with planar interfaces, currently all conductors must be on
one side of the interface. 

\end{itemize}

\subsection{Warnings and Errors}

\thecode introduces two new warnings the user may encounter. 

\begin{verbatim}
placeq: Warning, a panel is larger than the cube supposedly containing it
\end{verbatim}\vspace{-\topsep}

The number of grid points in the domain is large enough such that a basic
computational cell is smaller than some panel in the domain.  
If $m$ is the grid level specified by {\tt -d$m$}, then the grid spacing
$\Delta = L / (2^m-1)$, where $L$ is the length of the problem domain along
the longest axis.   If $p$ is the grid order specified by the {\tt -G$p$}
flag the condition  

\[ R < (p-1) \Delta \] 

must be satisfied within ten percent or so, or the results produced by
\thecode may be inaccurate.  Convergence of the iterative solver may also
be impaired. 

\begin{verbatim}
Warning: a panel is in the wrong half-space; results may be invalid
\end{verbatim}\vspace{-\topsep}

If a planar interface at $z$-coordinate $z_0$ is specified with the flag
{\tt -z$z_0$}, all panel $z$-coordinates must be greater than $z_0$ or this
warning results.   Behavior of the code may be unpredictable. 

\subsection{Bugs}

\thecode has several shortcomings, some of which are itemized here. The
user may report other bugs by sending mail 
to \verb~fftcap-bug@rle-vlsi.mit.edu~.

\begin{itemize}

\item There is no preconditioner. 

\item Non-planar or multiple dielectric interfaces are not implemented. 

\item Cell sizes smaller than a panel size, due to an excessively fine
  grid, may lead to substantial errors in the computed capacitances.

\item The {\tt -p} option does not work currently.  

\item Changing planar interface capability requires recompilation. 

\item \thecode does not work well with very imhomogeneous discretizations. 

\end{itemize}

\section{Compiling}
\label{comfas}

To enable the analysis of planar interfaces, the compile flag {\tt GNDPLN}
in the file {\tt src/mulGlobal.h} should be set {\tt ON}.  The code will
then assume the presence of planar interfaces.  For execution on problems
without interfaces, the flag should be set {\tt OFF}. 

\end{document}
